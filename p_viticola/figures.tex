% \documentclass[english, draft]{article}
\documentclass[english]{article}

%\usepackage[showframe]{geometry}
\usepackage{geometry}
\usepackage{float}
\usepackage[utf8]{inputenc}
\usepackage[backend=biber,style= authoryear]{biblatex}
% \usepackage[backend=biber]{biblatex}
\usepackage[english]{babel}
\usepackage{csquotes}
\usepackage{graphicx}
\usepackage{subcaption}
\usepackage{booktabs}
\usepackage{xargs}
\usepackage[pdftex,dvipsnames,table]{xcolor}
\usepackage{bbold}

\graphicspath{{./resources/images}}
\addbibresource{../articles.bib}
\addbibresource{../manual.bib}

\geometry{a4paper, total={165mm,257mm}, left=15mm, top=20mm,}

\begin{document}
\pagenumbering{arabic}

\begin{abstract}

\end{abstract}

\section{Background}

\section{Material and Method}

\begin{figure}[H]
    \begin{center}
        \includegraphics[width=0.5\linewidth]{2023_a_oiv_imaging_system.jpg}
        \caption{VEGOIA platform's manual imaging system}\label{fig:vegoia}
    \end{center}
\end{figure}

\begin{figure}[H]
    \centering
    \begin{subfigure}[b]{0.3\linewidth}
        \includegraphics[width=\linewidth]{oiv1.png}
        \caption{OIV 1}\label{fig:oiv1}
    \end{subfigure}
    \begin{subfigure}[b]{0.3\linewidth}
        \includegraphics[width=\linewidth]{oiv3.png}
        \caption{OIV 3}\label{fig:oiv3}
    \end{subfigure}
    \begin{subfigure}[b]{0.3\linewidth}
        \includegraphics[width=\linewidth]{oiv5.png}
        \caption{OIV 5}\label{fig:oiv5}
    \end{subfigure}
    \begin{subfigure}[b]{0.3\linewidth}
        \includegraphics[width=\linewidth]{oiv7.png}
        \caption{OIV 7}\label{fig:oiv7}
    \end{subfigure}
    \begin{subfigure}[b]{0.3\linewidth}
        \includegraphics[width=\linewidth]{oiv9.png}
        \caption{OIV 9}\label{fig:oiv9}
    \end{subfigure}
    \caption{Examples of leaf discs annotated with OIV 452-1 values. Levels increase with pathogen resistance. Upper row, susceptible leaf discs with levels 1 to 5. Lower row, resistant leaf disc~\ref{fig:oiv7} and fully resistant leaf disc~\ref{fig:oiv9}}\label{fig:phenotypes}
\end{figure}

\begin{figure}[H]
    \begin{center}
        \includegraphics[width=0.9\linewidth]{2023_a_oiv_indexation}
        \caption{Leaf disc extraction workflow}\label{fig:preprocessing}
    \end{center}
\end{figure}

\begin{figure}[H]
    \begin{center}
        \includegraphics[width=0.7\linewidth]{2023_a_oiv_oiv_distribution}
        \caption{Visualization of the split of the annotated data set for training models}\label{fig:datadistribution}
    \end{center}
\end{figure}

\begin{figure}[H]
    \centering
    \includegraphics[width=0.9\linewidth]{p_viticola/resources/images/swin_transformer.png}
    \caption{Swin Transformer architecture to serve as backbone for the ordinal regression methods to quantify the symptoms in leaf discs}
    \label{fig:enter-label}
\end{figure}

\begin{equation}
    -\sum_{c=1}^My_{o,c}\log(p_{o,c})\label{fml:crossentropy}
\end{equation}

\begin{equation}
    f(x^{[i]}) = \hat{P}(y^{[i]} > r_{k}|y^{[i]} > r_{k-1})\label{fml:binclass}
\end{equation}

\begin{equation}
    \hat{P}(y^{[i]} > r_{k}) = \prod_{j=1}^{k}f_{j}(x^{[i]})\label{fml:unconditionalprob}
\end{equation}

\begin{equation}
    q^{[i]} = 1 + \sum_{j=1}^{K-1}\mathbb{1}(\hat{P}(y^{[i]} > r_{j}) > 0.5)\label{fml:rankprob}
\end{equation}

\begin{itemize}
    \item Cross entropy loss~\ref{fml:crossentropy}
    \item Binary classification conditional probability~\ref{fml:binclass}
    \item Unconditional probabilities~\ref{fml:unconditionalprob}
    \item Rank prediction~\ref{fml:rankprob}
\end{itemize}

\section{Results and Discussion}


\begin{table}[H]
    \centering
    \caption{Average performance metrics with different ordinal regression modes}
    \label{tab:dtafracevol}
    \begin{tabular}{llrrrr}
        \toprule
        Ordinal regression & backbone   & accuracy    & f1 weighted avg & MSE         & MAE         \\
        \midrule
        Naive              & pretrained & 0.775±0.016 & 0.769±0.021     & 0.229±0.018 & 0.227±0.016 \\
        % CORAL              & pretrained & 0.447±0.027 & 0.368±0.055     & 0.737±0.052 & 0.612±0.024 \\
        CORN               & pretrained & 0.778±0.014 & 0.772±0.017     & 0.224±0.015 & 0.223±0.014 \\
        \bottomrule
    \end{tabular}
\end{table}


\begin{table}[H]
    \centering
    \caption{Confusion matrix precision recall and F1-score where the best ordinal regression method is used, the CORN approach is used}
    \label{tab:dtafracevol}
    \begin{tabular}{lrrrrrrrr}
        \toprule
        \textbf{True OIV} & \textbf{Predicted} &            &            &            &            & \textbf{F1-score}   \\
        {}                & \textbf{1}         & \textbf{3} & \textbf{5} & \textbf{7} & \textbf{9} &                   & \\
        \midrule
        \textbf{1}        & 48                 & 2          & 0          & 0          & 0          & 0.91                \\
        \textbf{3}        & 7                  & 32         & 4          & 0          & 0          & 0.81                \\
        \textbf{5}        & 0                  & 2          & 38         & 7          & 0          & 0.84                \\
        \textbf{7}        & 0                  & 0          & 2          & 78         & 12         & 0.77                \\
        \textbf{9}        & 0                  & 0          & 0          & 26         & 25         & 0.57                \\
        \bottomrule
    \end{tabular}
\end{table}

% \begin{table}[H]
%     \centering
%     \caption{Confusion matrix precision recall and F1-score where the best ordinal regression method is used, the CORN approach is used}
%     \label{tab:dtafracevol}
%     \begin{tabular}{lrrrrrrrr}
%         \toprule
%         \textbf{True OIV} & \textbf{Predicted} &            &            &            &            & \textbf{F1-score}   \\
%         {}                & \textbf{1}         & \textbf{3} & \textbf{5} & \textbf{7} & \textbf{9} &                   & \\
%         \midrule
%         \textbf{1}        & 41                 & 7          & 0          & 0          & 0          & 0.90                \\
%         \textbf{3}        & 2                  & 36         & 5          & 0          & 0          & 0.83                \\
%         \textbf{5}        & 0                  & 1          & 43         & 3          & 0          & 0.86                \\
%         \textbf{7}        & 0                  & 0          & 5          & 69         & 18         & 0.77                \\
%         \textbf{9}        & 0                  & 0          & 0          & 15         & 36         & 0.69                \\
%         \bottomrule
%     \end{tabular}
% \end{table}


\section{Conclusions and perspectives}


\begin{figure}[H]
    \centering
    \begin{subfigure}[b]{0.45\linewidth}
        \includegraphics[width=\linewidth]{water.png}
        \caption{Water droplets may be mistaken with sporulation}\label{fig:error97water}
    \end{subfigure}
    \begin{subfigure}[b]{0.45\linewidth}
        \includegraphics[width=\linewidth]{error_97.png}
        \caption{White stains appear in image}\label{fig:error97b}
    \end{subfigure}
    \caption{Images with OIV value 9 predicted as 7}\label{fig:errors97}
\end{figure}

\begin{figure}[H]
    \centering
    \begin{subfigure}[b]{0.45\linewidth}
        \includegraphics[width=\linewidth]{error_79_2.png}
        \caption{}\label{fig:error79a}
    \end{subfigure}
    \begin{subfigure}[b]{0.45\linewidth}
        \includegraphics[width=\linewidth]{error_79_1.png}
        \caption{}\label{fig:error79b}
    \end{subfigure}
    \caption{Images with OIV value 7 predicted as 9 when sporulation is hardly perceptible}\label{fig:errors79}
\end{figure}

\textbf{Sabine}
- Améliorer qualité images
- Eliminer bruit
- améliorer résolution pour améliorer annotations et modèles
- Prédire QTL
- Entrainer modèle sur images propres
- Tester
- reste dataset
- QTLs ?
- Tester sur pop avec QTLs connus ?
Tester sur exp qu n'a pas produit des QTLs


\end{document}